%&LaTeX
% !TEX encoding = UTF-8 Unicode
\documentclass{article}
\usepackage[utf8]{inputenc}
\usepackage[francais]{babel}
\usepackage[T1]{fontenc}
\usepackage{textcomp}
\usepackage[colorlinks=true, linkcolor=black, urlcolor = blue]{hyperref}

\usepackage{ulem}
\usepackage{color}
\usepackage[top=2.5cm,bottom=3cm,left=3cm,right=3cm]{geometry}


\definecolor{color01}{rgb}{0.00,0.00,0.00}
\definecolor{color02}{rgb}{0.40,0.40,0.40}
\definecolor{color03}{rgb}{0.07,0.33,0.80}

\title{\Huge{AGETAC\\
Aide à la GEstion TACtique\\}
\huge{{\color{color02} \textit{Rapport Post-Mortem}}}}

\date{Mai 2012}
\author{Projet Agetac - Université de Rennes 1\\
Encadré par Noël Plouzeau}

\begin{document}

\maketitle

%\vspace{138pt}
%{\color{color01} Ce document doit répondre aux questions suivantes:}
%
%
%{\color{color01} qu'est-ce qui n'a pas marché lors de la gestion de ce projet, 
%et pourquoi? comment cela aurait-il pu être évité?}
%
%{\color{color01} qu'est-ce qui a bien marché, et pourquoi? que faudra-t-il faire 
%pour reproduire ce résultat?\label{h.rzqw1zvcpnvw}}

\vspace{1in}
\tableofcontents
\newpage

\section{Introduction}

Dans ce rapport, nous étudierons plusieurs points du projet. Tout d’abord, nous parlerons des points importants qui ont bien marché, qui ne nous ont pas posé de problèmes particuliers lors de leurs mises en place. Ensuite, nous détaillerons les aspects du projet qui nous ont pris plus de temps ou qui n’ont pas fonctionné. Nous proposerons également une solution pour ces problèmes. Pour finir, nous reviendrons sur les fonctionnalités que nous n’avons pas eu le temps d’implémenter faute de temps, mais qui sont prévues pour les versions ultérieures du projet.

\section{Ce qui a fonctionné}


\subsection{Android}

Une fois la technologie Android prise en main, nous n’avons pas eu de problèmes majeurs quant à son utilisation. Le fait que les deux langages utilisés JAVA et XML soient deux langages dont nous connaissions déjà la syntaxe nous a facilité la tâche. La documentation disponible sur internet nous a aussi permis d’avancer rapidement.  

\subsection{OpenStreetMap}

OpenStreetMap est un service de cartographie communautaire permettant de récupérer facilement des cartes très précises. Ce service a été utilisé dans le projet Agetac plutôt que GoogleMap du fait des mises à jour plus régulières possible grâce à sa grande base d’utilisateurs et également de la politique de Google prévoyant une contribution pour un grand nombre d’utilisation. Étant un ensemble libre mais sous licence, nous créditons donc OpenStreetMap via ce lien : \url{http://www.openstreetmap.org/}

\subsection{RESTlet and Jackson}

Restlet est un framework Java complet et léger qui nous permet de mettre en place une architecture web (REST) et de bénéficier de sa simplicité. Restlet possède un noyau léger mais grâce à son architecture basé sur les plugins, c’est aussi un framework REST complet pour Java. Il supporte tous les concepts REST (Ressources, Représentions, Connecteur, Composants, etc.) et nous l'utilisons côté serveur et client.

Pour la sérialisation des objets, nous utilisons Jackson, une librairie Java multi-usage pour manipuler le format JSON. 

\subsection{Persistance}

Dans le serveur Agetac, nous utilisons les Java Data Objects (JDO), une interface standard pour stocker les objets dans une base de données. Des interfaces pour annoter des objets Java, récupérer des objets grâce a des requêtes, et interagir avec une base de données en utilisant des transactions sont définies dans la librairie standard. Une application utilisant les interfaces JDO peut fonctionner avec différents types de base de données sans utiliser de code spécifique, incluant les bases de données relationnelle, les bases de données hiérarchique, et les bases de données d’objets. Comme d’autres interfaces standards, JDO simplifie le portage du serveur d’Agetac vers d’autres solutions de stockage.

Des difficultés ont été rencontrées, elles sont décrites dans le Document de développement.

\newpage
\section{Ce qui n'a pas fonctionné}


\subsection{Authentification}
Nous avons tenté de mettre en place un système d’authentification pour sécuriser les échanges de données et permettre l’authentification des utilisateurs mais le système ne fonctionnait pas correctement et par la suite, de nombreux changements ont été effectués et par manque de temps, nous avons préféré laisser cette fonctionnalité de coté.

Il n’y a donc actuellement pas de système d’authentification efficace en place pour ce qui est de l’accès à l’application sur les tablettes. Dans le futur, les agents possédant une tablette devront s'authentifier pour accéder aux interventions ou en créer. Cela permettra de sécuriser les informations et de mettre en place un système d’autorisation pour les différentes fonctionnalités de l’application.

\subsection{(Dé)sérialisation en JSON}

Nous avons tout d’abord implémenté nous même un système de conversion des objets en JSON mais de nombreuses erreurs nous ont poussés à utiliser la librairie Jackson. Ces problèmes nous ont fait perdre beaucoup de temps mais nous ont aussi obligés à repenser l’architecture du serveur dans son intégralité. Cette nouvelle architecture est plus robuste, modulaire et standardisée.


\section{Ce que nous n'avons pas eu le temps d'implémenter}

En raison d’un manque de temps, certaines fonctionnalités que nous avions prévu pour l’application lors de l’analyse n’ont pas pu être implémentées. Nous les rappelons ici en expliquant brièvement ce qu’elles auraient du apporter à Agetac.

\subsection{Gestion des agents}

Suite à une décision prise en accord avec le client, il n’y a pas de système de gestion des personnes dans la version actuelle. Les pompiers présents sur l’intervention sont identifiés par le véhicule auquel ils sont attachés. Si l’application doit prendre en compte cet aspect, ce ne sera que bien plus tard.

\subsection{Hiérarchie}

A tout moment, un utilisateur pourra consulter la hiérarchie de l’intervention indiquant le commandement, les groupes et les canaux radios par lesquels les différents chefs pourront être contactés. Étant donné que la gestion des agents n’a pas encore été mise en place, cette fenêtre n’est pas encore disponible.

\subsection{Sectorisation}

Du fait de quelques problèmes rencontrés avec la SITAC, nous n’avons pas eu le temps de mettre en place le système de sectorisation fonctionnelle et géographique. Celui-ci devra permettre au COS de regrouper les véhicules par secteurs, suivant les zones de dangers et les types de véhicules impliqués.

\subsection{Groupe }

Le regroupement des véhicules sous un même commandement n’est pas encore mis en oeuvre. A terme, il permettra au COS de regrouper des petits groupes de véhicules au sein d’un groupe, afin de faciliter les déplacements et la transmission des ordres.

\subsection{Historique}

Une fonctionnalité qui sera à implémenter est la notion d’historique d’intervention. Toutes les interventions qui se déroulent avec l’aide d’AGETAC seront enregistrées étapes par étapes. Cet historique aura un intérêt particulier pour la formation des Officiers Sapeurs-pompiers et leurs permettra de revoir pas à pas le déroulement d’une intervention.

Associé à cet historique, un outil de simulation d’intervention pourra être ajouté, toujours dans le cadre de la formation.

\subsection{Tableau des victimes}

L’application devra disposer d’un tableau recensant les victimes d’une intervention en fonction de leur état. Ce tableau sera visualisable par les agents possédant une tablette et sera modifiable par le COS.

\subsection{Réception des Messages du CODIS sur la tablette.}

L’envoi de messages par le CODIS sur les tablettes est possible actuellement cependant aucune distinction n’est faites entre un message qui à été envoyé de la tablette vers le CODIS et un message qui à été envoyé du CODIS vers la tablette. De plus nous avions prévu de mettre en place un système de notification lors de la réception d’un message de la part du CODIS, ceci n’a pas pu être implémenté à temps.


\section{Conclusion}
Pour conclure, on constate que nous n’avons pas eu de problèmes particuliers lors du déroulement du projet, excepté le problème de (dé)sérialisation JSON qui nous a obligé à repenser toute l’architecture du serveur. Le fait d’être aidé par une équipe de recherche nous a permis d’éviter de prendre trop de retard sur certains points difficiles. C’est un projet d’envergure et faute de temps, bien des fonctionnalités restent à implémenter. Cependant, les bases ont été posées et le projet va continuer à être développé pour devenir au final un outil important pour l’amélioration des interventions de pompiers.
\end{document}
