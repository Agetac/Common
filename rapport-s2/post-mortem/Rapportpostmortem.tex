%&LaTeX
% !TEX encoding = UTF-8 Unicode
\documentclass{article}
\usepackage[utf8]{inputenc}
\usepackage[T1]{fontenc}
\usepackage{textcomp}

\usepackage{ulem}
\usepackage{color}

\definecolor{color01}{rgb}{0.00,0.00,0.00}
\definecolor{color02}{rgb}{0.40,0.40,0.40}
\definecolor{color03}{rgb}{0.07,0.33,0.80}

\begin{document}
\baselineskip=13pt
{\color{color01} Rapport post mortem (Analyse de gestion de projet). }

\vspace{55pt}
\begin{center}
{\Huge {\color{color01} \textbf{AGETAC}}}

{\Huge {\color{color01} \textbf{Aide à la GEstion TACtique}}}

\vspace{41pt}
{\huge {\color{color02} \textit{Rapport Post-Mortem}}}
\end{center}

\vspace{138pt}
{\color{color01} Ce document doit répondre aux questions suivantes:}

\leftskip=36pt
\parindent=-18pt
{\color{color01} qu'est-ce qui n'a pas marché lors de la gestion de ce projet, 
et pourquoi? comment cela aurait-il pu être évité?}

{\color{color01} qu'est-ce qui a bien marché, et pourquoi? que faudra-t-il faire 
pour reproduire ce résultat?\label{h.rzqw1zvcpnvw}}

\vspace{51pt}
\section*{{\LARGE {\color{color01} \textbf{Sommaire}}}}

\vspace{13pt}
\leftskip=18pt
\parindent=0pt
{\color{color03} \emph{Sommaire}}

{\color{color03} \emph{Introduction}}

{\color{color03} \emph{Ce qui a marché}}

{\color{color03} \emph{Ce qui n'a pas marché}}

{\color{color03} \emph{Ce que nous n'avons pas eu le temps d'implémenter\label{h.hqop80fo4yag}\label{h.9rnlmycclih6}\label{h.dbnzf7me2hmh}}}

\vspace{106pt}
\section*{{\LARGE {\color{color01} \textbf{Introduction}}}}

\vspace{13pt}
\leftskip=0pt
{\color{color01} Dans ce rapport, nous étudierons plusieurs points du projet.}

{\color{color01} Tout d'abord, nous parlerons des points importants qui ont bien 
marché, qui ne nous ont pas posé de problèmes particuliers lors de leurs mises 
en place. Ensuite, nous détaillerons les aspects du projet qui nous ont pris plus 
de temps ou qui n'ont pas fonctionné. Nous proposerons également une solution 
pour ces problèmes. Pour finir, nous reviendrons sur les fonctionnalités que 
nous n'avons pas eu le temps d'implémenter faute de temps, mais qui sont prévues 
pour les versions ultérieures du projet.\label{h.mqlx3ywo6evw}}

\vspace{24pt}
\section*{{\LARGE {\color{color01} \textbf{Ce qui a marché }}}}

\vspace{13pt}
\parindent=3pt
{\large {\color{color01} - Android}}

\vspace{13pt}
\parindent=0pt
{\color{color01} Une fois la technologie Android prise en main, nous n'avons pas 
eu de problèmes majeurs quant à son utilisation. Le fait que les deux langages 
utilisés JAVA et XML soient deux langages dont nous connaissions déjà la syntaxe 
nous a facilité la tâche. La documentation disponible sur internet nous a permis 
d'avancer rapidement.  }

\vspace{27pt}
{\large {\color{color01} - OpenStreetMap}}

\vspace{13pt}
{\color{color01} OpenStreetMap est un service de cartographie communautaire permettant 
de récupérer facilement des cartes très précises. Ce service a été utilisé 
dans le projet Agetac plutôt que GoogleMap du fait des mises à jour plus régulières 
possible grâce à sa grande base d'utilisateurs et également de la politique 
de Google prévoyant une contribution pour un grand nombre d'utilisation. Étant 
un ensemble libre mais sous licence, nous créditons donc OpenStreetMap via ce 
lien : }{\color{color03} \emph{http://www.openstreetmap.org/}}

\vspace{13pt}
\parindent=3pt
{\large {\color{color01} - RESTlet and Jackson}}

\vspace{13pt}
\parindent=0pt
{\color{color01} Restlet is a comprehensive yet lightweight RESTful web framework 
for Java that allowed us to embrace the architecture style of the Web (REST) and 
benefit from its simplicity. Restlet has a light core but thanks to its plugin 
architecture, it is also a comprehensive REST framework for Java. It supports all 
REST concepts (Resource, Representation, Connector, Component, etc.) and we use 
it both in the Client and the Server side.}

\vspace{13pt}
{\color{color01} For (de)serialization we use Jackson. Jackson is a multi-purpose 
Java library for processing JSON data format. It }{\color{color01} \textit{just 
works}}{\color{color01}  and it is very intuitive to work with.}

\vspace{13pt}
\parindent=3pt
{\large {\color{color01} - Persistence}}

\vspace{13pt}
\parindent=0pt
{\color{color01} In the Agetac server we use Java Data Objects (JDO), a standard 
interface for storing objects containing data into a database. The standard defines 
interfaces for annotating Java objects, retrieving objects with queries, and interacting 
with a database using transactions. An application that uses the JDO interface 
can work with different kinds of databases without using any database-specific 
code, including relational databases, hierarchical databases, and object databases. 
As with other interface standards, JDO simplifies porting the Agetac Server between 
different storage solutions.}

\vspace{13pt}
{\color{color01} Some difficulties were encountered, the specifics are described 
in the }{\color{color01} \textit{Document de développement\label{h.ssdrr26ojh9l}}}

\vspace{24pt}
\section*{{\LARGE {\color{color01} \textbf{Ce qui n'a pas marché}}}}

\parindent=3pt
{\large {\color{color01} - Authentification\label{h.ifmwcz4cx8ke}}}

\vspace{37pt}
\section*{{\LARGE {\color{color01} \textbf{Ce que nous n'avons pas eu le temps 
d'implémenter}}}}

\vspace{27pt}
\parindent=0pt
{\color{color01} En raison d'un manque de temps, certaines fonctionnalités que 
nous avions prévu pour l'application lors de l'analyse n'ont pas pu être implémentées. 
Nous les rappelons ici en expliquant brièvement ce qu'elles auraient du apporter 
à Agetac.}

\vspace{13pt}
{\large {\color{color01} - Authentification}}

{\color{color01} Actuellement, il n'y a pas de système d'authentification efficace 
en place pour ce qui est de l'accès à l'application sur les tablettes. Dans le 
futur, les agents possédant une tablette devront s'authentifier pour accéder 
aux interventions ou en créer. Cela permettra de sécuriser les informations et 
de mettre en place un système d'autorisation pour les différentes fonctionnalités 
de l'application.}

\vspace{13pt}
{\large {\color{color01} - Gestion des agents}}

{\color{color01} Suite à une décision prise en accord avec le client, il n'y 
a pas de système de gestion des personnes dans la version actuelle. Les pompiers 
présents sur l'intervention sont identifiés par le véhicule auquel ils sont 
attachés.}

\vspace{13pt}
\parindent=3pt
{\large {\color{color01} - Hiérarchie}}

\parindent=0pt
{\color{color01} A tout moment, un utilisateur pourra consulter la hiérarchie 
de l'intervention indiquant le commandement, les groupes et les canaux radios par 
lesquels les différents chefs pourront être contactés. Étant donné que la 
gestion des agents n'a pas encore été mise en place, cette fenêtre n'est pas 
encore disponible.}

\vspace{13pt}
{\large {\color{color01} - Sectorisation}}

{\color{color01} Du fait de quelques problèmes rencontrés avec la SITAC, nous 
n'avons pas eu le temps de mettre en place le système de sectorisation fonctionnelle 
et géographique. Celui-ci devra permettre au COS de regrouper les véhicules par 
secteurs, suivant les zones de dangers et les types de véhicules impliqués.}

\vspace{13pt}
\parindent=3pt
{\large {\color{color01} - Groupe }}

\parindent=0pt
{\color{color01} Le regroupement des véhicules sous un même commandement n'est 
pas encore mis en oeuvre. A terme, il permettra au COS de regrouper des petits 
groupes de véhicules au sein d'un groupe, afin de faciliter les déplacements 
et la transmission des ordres.}

\vspace{13pt}
{\large {\color{color01} - Historique}}

\parindent=36pt
{\color{color01} Une fonctionnalité qui sera à implémenter est la notion d'historique 
d'intervention. Toutes les interventions qui se déroulent avec l'aide d'Agetac 
seront enregistrées étapes par étapes. Cet historique aura un intérêt particulier 
pour la formation des Officiers Sapeurs-pompiers et leur permettra de revoir pas 
à pas le déroulement d'une intervention.}

\parindent=0pt
{\color{color01} Associé à cet historique, un outil de simulation d'intervention 
pourra être ajouté, toujours dans le cadre de la formation.}

\vspace{13pt}
{\large {\color{color01} - Tableau des victimes}}

{\color{color01} L'application devra disposer d'un tableau recensant les victimes 
d'une intervention en fonction de leur état. Ce tableau sera visualisable par 
les agents possédant une tablette et sera modifiable par le COS.}

\vspace{13pt}
{\color{color01} - Réception des Messages du CODIS sur la tablette.}

\end{document}
