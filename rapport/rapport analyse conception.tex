\documentclass{article}
\usepackage{verbatim}
\usepackage{graphicx}
\usepackage{amsmath}
\usepackage{stmaryrd}
\usepackage{amsfonts}
\usepackage[french]{babel}

\usepackage[utf8]{inputenc}
\title{Rapport d'analyse et de conception\\Projet AGETAC}
\date{20/12/2011}

\newcommand{\todo}[1]{\textcolor{green}{#1}}

\begin{document}

\maketitle

\tableofcontents 


\section*{Introduction}

Dans le cadre de notre première année de Master Informatique, nous devons réaliser un projet de l’analyse à l’intégration afin de nous familiariser aux mécaniques du développement logiciel.

Notre équipe travaille sur le projet AGETAC, une application d’aide à la gestion tactique pour les sapeurs-pompiers d’Ille-et-Vilaine. Cette application sera développée pour des tablettes tactiles fonctionnant sous le système Android. Elle sera accompagnée d’un serveur de gestion des moyens et de synchronisation des données entre les tablettes.

Ce document fait suite au rapport de présentation visant à détailler les tenants et aboutissants du projet. Il constitue une ressource technique de l’architecture de l’application AGETAC.

Ce rapport commencera par analyser les fonctionnalités attendues au moyen de diagrammes de cas d’utilisation et de séquences. Ces diagrammes représentent les actions que le système sera capable de prendre en charge dans sa première version. Afin d’avoir une vue globale du projet, que ce soit côté client ou serveur, nous avons réalisé deux diagrammes statiques d’analyse et de conception. Enfin, nous présenterons la stratégie de développement que nous avons adoptée : le cycle en spirale.


\section{Cas d'utilisation}

\section{Diagrammes de séquence}

\section{Stratégie de développement}

\section*{Conclusion}
A la fin de la première partie du projet, nous avons donc mis en place une bonne partie de l’analyse et de la conception d’une première version de notre logiciel.

Les différentes présentations de Noël Plouzeau, ainsi que l’intervention du Capitaine Régis Demay (des sapeurs-pompiers d’Ille et Vilaine) nous ont permis d’apprendre le fonctionnement d’une intervention des sapeurs pompiers et les conventions qu’ils utilisent pour communiquer, afin de produire un logiciel répondant au mieux aux besoins de leurs utilisateurs.

Nous avons de plus acquis une partie des connaissances nécessaires pour la partie programmation du projet. En effet, nous avons produit des POC (Proof of Concept), d’une part  avec première ébauche du client (pour laquelle il a fallu apprendre l’API d’Android), d’autre part pour la communication client/serveur (pour laquelle il a fallu apprendre à utiliser le framework RESTlet).

Il nous reste donc pour la suite du projet à implémenter et tester une première version du logiciel, puis ajouter les fonctionnalités supplémentaires qui seront demandées par le Capitaine Régis Demay sous forme de nouvelles itérations.

\end{document}
